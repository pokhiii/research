\section{Summary and Takeaways}

Polynomial Identity Testing (PIT) asks a deceptively simple question: Does a given formula always evaluate to zero? Despite its elementary appearance, this question sits at the heart of deep complexity theory questions like whether randomness is truly necessary for efficient computation.

We saw that PIT has fast randomized algorithms, made trustworthy by the Schwartz-Zippel Lemma. These algorithms use random inputs to probabilistically verify whether a formula is identically zero — with error probabilities that shrink exponentially with repetition.

However, deterministic (error-free) algorithms remain elusive in the general case. Exciting progress has been made in special cases — like read-once formulas, depth-3 circuits with bounded fan-in, and algebraic branching programs — where deterministic checks are now possible.

PIT is more than just a math puzzle. It’s a lens into the power of randomness, the structure of algebraic computation, and one of the most important open questions in computer science: \textbf{Can every randomized algorithm be derandomized?}

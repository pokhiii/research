\section{The Basic Question: Always Zero?}


\subsection*{What Does ``Always Equal to Zero'' Mean?}

Take this formula:
\[
(x + y)^2 - x^2 - 2xy - y^2
\]
Let’s plug in some numbers and see what we get.

\paragraph{Try 1 (x = 1, y = 2):}
\begin{align*}
(1 + 2)^2 - 1^2 - 2 \cdot 1 \cdot 2 - 2^2 &= 9 - 1 - 4 - 4 \\
&= 0
\end{align*}

\paragraph{Try 2 (x = -3, y = 5):}
\begin{align*}
(-3 + 5)^2 - (-3)^2 - 2 \cdot (-3) \cdot 5 - 5^2 &= 4 - 9 + 30 - 25 \\
&= 0
\end{align*}

\paragraph{Try 3 (x = 0, y = 0):}
\[
(0 + 0)^2 - 0^2 - 0 - 0 = 0
\]

No matter what numbers we choose, this big formula always gives $0$.

\medskip
This is what we mean when we say:

\textbf{``The formula is always equal to zero.''}

\subsection*{A Formula That’s Not Always Zero}

Take this one:
\[
(x + y)^2 - x^2 - y^2
\]

Try plugging in:

\paragraph{Try 1 (x = 1, y = 2):}
\begin{align*}
(1 + 2)^2 - 1^2 - 2^2 &= 9 - 1 - 4 \\
&= 4
\end{align*}

\paragraph{Try 2 (x = 0, y = 0):}
\[
(0 + 0)^2 - 0^2 - 0^2 = 0
\]

\medskip
Uh oh. Sometimes it’s $0$, sometimes not. So this formula is \textbf{not always zero}.

\bigskip
That’s what we’re testing in PIT:

\begin{quote}
\emph{Is this formula zero for all values of $x$ and $y$?}
\end{quote}

Even one counterexample is enough to say \textbf{No}.